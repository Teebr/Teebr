%!TEX root = rapport.tex

\section{Introduction}

Les moteurs de recommandation de contenu cherchent à trouver du contenu
pertinent en fonction de sa similarité avec du contenu consommé par un
utilisateur, tandis que les moteurs de recommandation de personnes sur les
réseaux sociaux (par exemple ami(e)s ou collègues) se basent principalement sur
le graph social et la place de chaque utilisateur dans celui-ci.

On s’intéresse ici au réseau social \twt{}, où chaque utilisateur peut produire
du contenu sous forme de cours messages (\tweets), lisibles par tous sur son
profil. Un utilisateur peut \I{s’abonner} à d’autres utilisateurs, ce qui lui
permet de recevoir le flux continu de tous les messages de ces abonnements sans
avoir à consulter chaque profil individuellement. Cette notion d’abonnement est
asymétrique : l’abonnement de A à B n’implique pas l’abonnement de B à A. Dans
le vocabulaire de la plateforme, on parle de \I{suivre} un utilisateur.

\twt{} est intéressant ici pour plusieurs points : le premier est la longueur
des messages. Ceux-ci sont très courts (140 caractères maximum), ce qui rend
l’extraction d’information plus difficile, car il n’y a pas de redondance et
beaucoup de termes sont abrégés. Le second est l’accès aux messages, qui sont
tous publics.\footnote{Il est possible de rendre un compte privé, mais peu
d’utilisateurs le font.} L’\api{} de \twt{} est par ailleurs très bien faite,
et il est facile de récupérer des messages, que l’on peut filtrer selon
plusieurs critères. Cela n’aurait pas été possible avec d’autres réseaux comme
\fb{}. Enfin, la qualité du flux que reçoit un utilisateur dépend des personnes
auxquelles il est abonné, donc la recommandation de contenu est fortement liée
à la recommandation d’utilisateurs.

Le projet \tb{} a pour but de faire de la recommandation d’utilisateurs sur
\twt{} non pas en fonction de leur place dans le graph social comme on le fait
habituellement, mais en fonction du contenu qu’ils produisent. On utilisera
pour cela un système de notation de \tweets{} pour construire le profil d’un
utilisateur.
