%!TEX root = rapport.tex

\section{Discussion}

\subsection{Problèmes rencontrés}

Plusieurs difficultés ont ralenti la réalisation de ce projet, mais c’est aussi
grâce à elles qu’un tel projet est intéressant.

\subsubsection{Exclusion du spam}

Nous avions très largement sous-estimé le spam sur \twt{}. Au début du projet,
les spams correspondaient à la majorité du contenu récupéré. Nous avons tenté
plusieurs techniques à base de règles sur les motifs repérés dans les spams,
mais beaucoup passaient entre les mailles du filet. La suppression de certains
mots-clefs a permis de supprimer certaines catégories de \tweets{}
indésirables. Par exemple, nous n’espérons pas extraire de \tweets{} sur le
langage de programmation \I{Swift} à cause de la chanteuse du même nom.

Les règles ajoutées à la main se révélant inefficaces, nous avons testé avec un
filtre Bayésien, et cette solution s’est révélée très efficace par rapport à la
précédente. Le programme utilise maintenant une combinaison des deux pour
supprimer la majorité du spam. L’application Web permet également de signaler
un \tweet{} comme étant du spam. Au bout d’un nombre fixe de signalements un
\tweet{} ainsi que son auteur est supprimé de la base.

\subsubsection{Catégorisation}

La catégorisation des \tweets{} s’annonçait déjà comme un problème difficile
avant le début de la réalisation. Nous n’avons pas à ce jour trouvé de solution
miraculeuse à ce problème. La longueur des messages et la faiblesse du
vocabulaire utilisé limitent très sérieusement les analyses linguistiques. Un
\tweet{} contient en moyenne 11 mots \citep{OConnor2010}, il n’y a donc pas de
répétition du sujet dans un message.

\subsection{Pistes non explorées}

Ce projet se concentre sur la recommandation basée sur les caractéristiques et
le contenu des \tweets{} rapportés à leur auteur. Nous aurions aussi pu tenter
de classifier les utilisateurs à partir de leurs \tweets~
\citep{Pennacchiotti11}. Une telle classification ne permet de trouver que des
profils sociaux, pas des catégories de contenu. Nous nous sommes également
concentrés sur l’extraction de mots individuels sans regarder les
co-occurrences de mots pour construire nos
catégories~\citep{Ramage2010,Rigouste06}. Cela aurait permis d’éviter les
ambiguités sur certains termes comme \I{Python} ou \I{Swift}. Enfin, nous ne
nous intéressons pas aux URLs contenues dans les \tweets{}, alors qu’il serait
également possible de les classifier, avec plus de succès ici car il existe de
nombreux corpus de catégorisation d’URLs, notamment en fonction du domaine du
site.
